\documentclass[letterpaper,12pt]{article}
\oddsidemargin 0cm
\evensidemargin 0cm
\textwidth 17 cm
\topmargin 0cm
\textheight 22.1 cm
\unitlength=1cm

%%%%Pacotes%%%%%%%%%
\usepackage[brazil,portuges]{babel}
\usepackage[latin1]{inputenc}
\usepackage{amssymb,amscd,latexsym,amsthm}
%\usepackage{color}
%\usepackage{pstricks}
%\usepackage{ulem}
%\usepackage{ifsym}
\usepackage{amsfonts}
\usepackage{amsmath}
%\usepackage{pslatex}
\usepackage{pxfonts}
%\usepackage{graphicx}
%\usepackage{wrapfig}
%\pagestyle{empty}
%\input{setbmp}






\def\dd{\displaystyle}
\newcommand{\mc}[1]{{\mathcal #1}}
\newcommand{\mf}[1]{{\mathfrak #1}}
\newcommand{\mb}[1]{{\mathbf #1}}
\newcommand{\bb}[1]{{\mathbb #1}}
\newcommand{\bs}[1]{{\boldsymbol #1}}
\def\p{\partial}


\begin{document}

\begin{center}
\textbf{\large{Limite Hidrodin\^amico}}\\
\end{center}

Caso simples.\\

\begin{itemize}
 \item $\rho:\mathbb {T}\rightarrow [0,1]$ cont\'inua. $\mathbb{T}$ \'e
o toro identificando o zero com o um.
\item Divide o toro em $N$ s\'itios, N bem grande. Sorteia se h\'a part\'icula no
k-\'esimo s\'itio segundo bernoulli($\rho(\frac{k}{N})$).
\item Cada s\'itio tem rel\'ogio exponencial independente. Quando toca, a 
part\'icula joga moeda honesta e olha para o lado. Se n\~ao h\'a part\'icula,
ela pula. 
\item Olha agora a configura\c c\~ao final no tempo $N^2t$. Repete a simula\c c\~ao
v\'arias vezes e constr\'oi a fun\c c\~ao que em cada s\'itio vale o n\'umero de vezes que apareceu part\'icula
no final dividido pelo n\'umero de simula\c c\~oes.\\
\end{itemize}


Generaliza\c c\~oes:\\

\begin{itemize}
 \item Escala de tempo pode ser $N^\alpha t$.
\item Mais part\'iculas por s\'itio.
\item O par\^ametro do rel\'ogio depende da configura\c c\~ao toda, ou seja, se o processo est\'a
em $\eta\in \bb{N}^{\bb{T}}$, o par\^ametro do rel\'ogio \'e $f(\eta)$. Quando o rel\'ogio toca, a configura\c c\~ao muda, com um salto de uma part\'icula (n\~ao precisa mais de ser para vizinho). A escolha $\eta\mapsto\eta^{x,y}$, onde $\eta^{x,y}$
significa que uma part\'icula pulou do s\'itio $x$ para o s\'itio $y$, \'e proporcional a 
$p(x,y)g(\eta)$. Em geral $g$ depende s\'o de $\eta(x)$, $\eta(y)$ e vizinhos de $x$ e $y$.\\
\end{itemize}

Caso que eu sugiro para teste, o resultado tem comportamento simples de ver.\\

\begin{itemize}
 \item O rel\'ogio depende s\'o do n\'umero total de part\'iculas e o par\^ametro \'e
$\dd\sum_x \eta(x)$. 
\item N\'umero de part\'iculas em cada s\'itio \'e livre.
\item A escala de tempo \'e $Nt$. Aten\c c\~ao, esse \'e $Nt$ mesmo, ao contr\'ario do $N^2t$ anterior.
\item Quando toca o rel\'ogio, a probabilidade com que $\eta\mapsto\eta^{x,x+1}$ \'e proporcional a $p\eta(x)$a probabilidade com que $\eta\mapsto\eta^{x,x+1}$ \'e proporcional a $(1-p)\eta(x)$ (passeios aleat\'orios independentes
assim\'etricos). 
\item A $\rho$ final \'e solu\c c\~ao da equa\c c\~ao do transporte, ou seja, a $\rho$ final \'e
simplesmente a $\rho$ inicial translada pra direita (se $p>\frac{1}{2}$) ou esquerda (se $p<\frac{1}{2}$).
\item Se $p=\frac{1}{2}$ a $\rho$ final \'e a $\rho$ inicial.
\end{itemize}




\end{document}







